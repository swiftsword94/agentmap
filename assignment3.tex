\documentclass[11pt, oneside]{article}   	% use "amsart" instead of "article" for AMSLaTeX format
\usepackage{geometry}                		% See geometry.pdf to learn the layout options. There are lots.
\geometry{letterpaper}                   		% ... or a4paper or a5paper or ... 
\usepackage{graphicx}				% Use pdf, png, jpg, or eps§ with pdflatex; use eps in DVI mode
								% TeX will automatically convert eps --> pdf in pdflatex		
\usepackage{amssymb}
\usepackage{program}

\title{Assignment 3: Bayesian Inference, Temporal State Estimation and Decision Making under Uncertainty}
\author{Alex Smirnov, Scott Reyes}
\date{April 11, 2017}							% Activate to display a given date or no date

\begin{document}
\maketitle
%\section{}
%\subsection{}
\begin{flushleft}

\section*{Problem 1:}
\subsection*{a}
The probability that all five of the Boolean variables are simultaneously true is:\\

$P(A)=0.2$\\
$P(B)=0.5$\\
$P(C)=0.8$\\
$P(D \mid A \wedge B)=0.1$\\
$P(E \mid B \wedge C)=0.3$\\ 
$P(A \wedge B)=0.1$\\
$P(A \wedge B \wedge C)=0.08$\\
$P(A \wedge B \wedge C)\times P(D \mid A \wedge B)=0.008$\\
$P(A \wedge B \wedge C)\times P(D \mid A \wedge B) \times P(E \mid B \wedge C)=0.0024$\\
\subsection*{b}
The probability that all five of the Boolean variables are simultaneously false is:\\
$P(\neg A)=0.8$\\
$P(\neg B)=0.5$\\
$P(\neg C)=0.2$\\
$P(\neg D \mid \neg A \wedge \neg B)=0.1$\\
$P(\neg E \mid \neg B \wedge \neg C)=0.8$\\
$P(\neg A \wedge \neg B)=0.4$\\
$P(\neg A \wedge \neg B \wedge \neg C)=0.08$\\
$P(\neg A \wedge \neg B \wedge \neg C) \times P( \neg D \mid \neg A \wedge \neg B)=0.008$\\
$P(\neg A \wedge \neg B \wedge \neg C)\times P(\neg D \mid \neg A \wedge \neg B) \times P(\neg E \mid \neg B \wedge \neg C)=0.0064$\\
\subsection*{c}
$P(\neg A)=0.8$\\
$P(D \wedge B)=0.7$\\
$P(D \wedge B \mid \neg A)=0.6$\\
$P(\neg A \mid D \wedge B)=\frac{0.8*0.6}{0.7}=0.686$\\\medskip
\section*{Problem 2:}
\subsection*{a}
Query: $P(Burglary \mid JohnCalls = true, MaryCalls = true)$\\
Variable Elimination\\
Query expression:\\
$P(B \mid j,m) = \alpha f_1 (B) * $
$\sum_{e}
f_2 (E) * $
$\sum_{a}
f_3 (A, B, E) * f_4 (A) * f_5 * (A)$\\

$f_6 (B,E)=$
$\sum_{a}
f_3 (A,B,E)*f_4 (A)*f_5 (A)$\\
$=(f_3 (a, B, E)* f_4 (a)* f_5 (a)) + (f_3 (\neg a, B, E)* f_4 (\neg a)* f_5 (\neg a))$

$P(B \mid j,m)= \alpha f_1 (B)*$
$\sum_{e}
f_2 (E) * f_6 (B,E)$

$f_7 (B)=$
$\sum_{e}
f_2 (E) * f_6 (B,E)$\\
$=f_2 (e) *f_6 (B,E)+f_2 (\neg e) * f_6 (B, \neg e)$

$P(B \mid j,m)= \alpha f_1 (B) * f_7 (B)$

\subsection*{b}
Variable Elimination Algorithm - Arithmetic Operations Performed\\
Additions: 1\\
Multiplications: 5\\
Divisions: 1\\
Tree Enumeration Algorithm - Operations Performed\\
Additions: 3\\
Multiplications: 9\\
Divisions: 1\\
\subsection*{c}
If a Bayesian network has the form of a chain, the complexity of computing $P(X_1 \mid X_n = true)$ using enumeration is $O(n)$ because every single X would need to be used in the calculation.\\
Computing the complexity with variable elimination is also $O(n)$ because no variables would be eliminated. Since the Bayesian network is a chain, there would be no eliminated variables and the complexity would be exactly the same as tree enumeration.\\
\section*{Problem 3:}
\subsection*{a}
\subsection*{b}
\subsection*{c}
\section*{Problem 4:}
\subsection*{a}

\subsection*{b}
\subsection*{c}
\subsection*{d}
\section*{Problem 5 - Programming Component:}
\subsection*{a}
\subsection*{b}
\subsection*{c - Generating Ground Truth Data}
\subsection*{d - Filtering and Viterbi Algorithms in Large Maps}
\subsection*{e}
\subsection*{f}
\subsection*{g}
\subsection*{h - Computational Approximations}
\end{flushleft}
\end{document}  